\chapter{Introdução}

Esse relatório apresenta uma análise de complexidade empírica para diferentes algoritmos de busca. Problemas de busca em um arranjo sequencial se resumem em procurar um dado valor chave $k$ em um conjunto de valores previamente armazenados em um arranjo $V$ passado como entrada do problema. Caso o valor seja encontrado neste arranjo, então o programa deve retornar o índice da localização de $k$ em $V$, caso contrário deve retornar $-1$. Este estudo está interessado em problemas de busca em que os elementos do arranjo estão ordenados em {\it ordem crescente} e são analisados apenas o pior cenário da busca. São considerados os seguintes algoritmos de busca: busca linear; busca binária (versões iterativas e recursiva); busca ternária (versões iterativas e recursiva); {\it jump search}; e busca de Fibonacci.

O primeiro objetivo desse estudo é determinar qual dos dois algoritmos lineares selecionados são mais eficientes (a busca linear ou a {\it jump search}). O segundo objetivo é determinar qual implementação é mais eficiente, a recursiva ou iterativa. O terceiro, é determinar como o tamanho da partição influência nos algoritmos de busca não lineares. O quarto é determinar a partir de que momento algoritmos de classe de complexidade diferentes se diferenciam, comparando a busca linear com a binária. Por fim, o quinto objetivo procura determinar se existe diferentes categorias de cenários de pior caso para o algoritmo de busca de Fibonacci.
