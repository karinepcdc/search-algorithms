\chapter{Introdução}

Esse relatório descreve uma análise de complexidade empírica de diferentes algoritmos de busca. São considerados os seguintes algoritmos de busca: busca linear; busca binária (versões iterativas e recursiva); busca ternária (versões iterativas e recursiva); {\it jump search}; e busca de Fibonacci.

O primeiro objetivo desse estudo é determinar qual dos dois algoritmos lineares são mais eficientes (a busca linear ou a {\it jump search}). O segundo objetivo é determinar qual implementação é mais eficiente, a recursiva ou iterativa. O terceiro, é determinar como o tamanho da partição influência nos algoritmos de busca não lineares. O quarto é determinar a partir de que momento algoritmos de classe de complexidade diferentes se diferenciam (comparando a busca linear com a binária). Por fim, o quinto objetivo procura determinar se existe diferentes categorias de cenários pior caso para o algoritmo de busca de Fibonacci.
