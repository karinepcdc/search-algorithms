\chapter{Resultados}

Nesta seção são analidos os resultados obtidos na seção anterior a fim de responder as perguntas que motivaram esse estudo.

\section{Algoritmos linear mais eficiente}

Determinar qual dos dois algoritmos lineares selecionados são mais eficientes (a busca linear ou a {\it jump search}).


\section{Implementação mais eficiente: recursiva ou iterativa?}

O segundo objetivo é determinar qual implementação é mais eficiente, a recursiva ou iterativa.



\section{Influência do tamanho da partição nos algoritmos de busca não lineares}

O terceiro, é determinar como o tamanho da partição influência nos algoritmos de busca não lineares.


\section{Diferenciação de algoritmos de classe de complexidade diferentes}

O quarto é determinar a partir de que momento algoritmos de classe de complexidade diferentes se diferenciam, comparando a busca linear com a binária.


\section{Cenários do algoritmo de Fibonacci}

Por fim, o quinto objetivo procura determinar se existe diferentes categorias de cenários de pior caso para o algoritmo de busca de Fibonacci.
