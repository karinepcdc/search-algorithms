\chapter{Conclusão}

Uma análise de complexidade empírica para os algoritmos de busca linear, busca binária (versões iterativas e recursiva), busca ternária (versões iterativas e recursiva), {\it jump search} e busca de Fibonacci foi feita. As simulações desses algoritmos consideravam apenas o pior cenário da busca em um arranjo com seus elementos ordenados em {\it ordem crescente}.

Com este estudo, determinou-se que entre os algoritmos de busca linear o {\it jump search} é o mais eficiente. Também foi possível verificar que a implementação iterativa é ligeiramente mais eficiente que a recursiva para tamanhos maiores de sistema. Comparando os algoritmos de busca binária, ternária e de Fibonacci, foi possível ver que o tamanho da partição influencia na busca e que as buscas ternária e binária são mais consideravelmente rápidas do que a busca de Fibonacci, sendo a busca ternária a mais eficiênte. Ademais, ao comparar a busca binária e linear, algoritmos de classe de complexidade diferentes, constatou-se que a diferenciação entre esse algoritmo ocorre já com tamanhos de arranjo pequenos (da ordem de 100). Além disso, verificou-se que existe duas categorias de cenários de pior caso para o algoritmo de Fibonacci, quando o valor procurado está fora do arranjo, mas mais próximo dos elementos do começo, o algoritmo tem uma performace ligeiramente melhor do que quando esse se encontra fora do arranjo, mas mais próximo dos elementos do fim.

Por fim, após essa análise foi possível concluir que os algoritmos mais eficiêntes são os de busca ternária e binária iterativos. Mas, como estes dependem do arranjo estar ordenado, em situações mais gerais, o único algoritmo disponível, dentre os estudados, é o de busca linear.
